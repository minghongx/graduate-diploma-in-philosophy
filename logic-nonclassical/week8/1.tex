\begin{enumerate}[label=(\roman*)]
\item
Let $M = \langle D, I \rangle$ be the model such that $D = \{m, n\}$ and $I$ defined as
\begin{center}
\begin{tblr}{
  colspec={c|c|c},
}
& $I(F)$ & $I(G)$ \\
\hline[solid]
$m$ & $1$ & $0$ \\
$n$ & $0$ & $1$ \\
\end{tblr}
\end{center}
Let $v$ be an assignment whose two x-variant are $v^x_m$ and $v^x_n$. This provides a counterexample to $\exists x Fx, \exists x Gx \Yright \exists x (Fx \land Gx)$.

The assignment $v^x_m$ yields
\begin{align}
\begin{split} \label{eq:1}
I(Fx,v^x_m) &= I(F)(I(x,v^x_m)) \\
&= I(F)(v^x_m(x)) \\
&= I(F)(m) \\
&= 1
\end{split} \\
I(Gx,v^x_m) &= \cdots = 0 \label{eq:2}
\end{align}

The assignment $v^x_n$ yields
\begin{align}
I(Fx,v^x_n) &= \cdots = 0 \label{eq:3} \\
I(Gx,v^x_n) &= \cdots = 1 \label{eq:4}
\end{align}

Since $v^x_m \sim_x v$ and $I(Fx,v^x_m) = 1$, $I(\exists x Fx, v) = 1$. Since $v^x_n \sim_x v$ and $I(Gx,v^x_n) = 1$, $I(\exists x Gx, v) = 1$. Thus, $I(A, v) = 1$ for each $A \in \{ \exists x Fx, \exists x Gx \}$, i.e., the premises are \textit{true}.

Since $I(Fx,v^x_m) = 1$ and $I(Gx,v^x_m) = 0$, $I(Fx \land Gx, v^x_m) = 0$. Similarly, $I(Fx \land Gx, v^x_n) = 0$. For every assignment $u \sim_x v$, $I(Fx \land Gx, u) = 0$, so $I(\exists x (Fx \land Gx), v) = 0$, i.e., the conclusion is \textit{false}.

\item
\begin{prooftree*}
\hypo{\exists x Fx}
\hypo{[Fa]^1}
\hypo{\forall x Gx}
\infer1[\une]{Ga}
\infer2[\ai]{Fa \land Ga}
\infer1[\exi]{\exists x (Fx \land Gx)}
\infer2[\exe{1}]{\exists x (Fx \land Gx)}
\end{prooftree*}

\item
Suppose the argument has a counterexample, setting $I(\exists x (Fx \to \forall y Fy), v) = 0$. Then, there is an object $o \in D$ such that $I(Fx \to \forall y Fy, v^x_o) = 0$. So, $I(Fx, v^x_o) = 1$ and $I(\forall y Fy, v^x_o) = 0$. The former implies $I(F)(o) = 1$. The latter implies for some $u \sim_y v^x_o$, $I(Fy, u) = 0$. The counterexample needs to satisfy these two conditions. We can construct a counterexample as follows:
\begin{center}
\begin{tblr}{
  width=0.5\textwidth,
  colspec={X[c]X[c]},
}
  $D = \{ m, n\}$
  &
  \begin{tblr}{c|c}
    & $I(F)$ & \\
    \hline[solid]
    $m$ & $1$ \\
    $n$ & $0$ \\
  \end{tblr}
\end{tblr}
\end{center}
Set arbitrary object $o$ to $m$. $I(Fx, v^x_m) = 1$ satisfies the first condition, $I(F)(m) = 1$. \\ $I(Fy, (v^x_m)^y_n) = 0$ satisfies the second: for some $u \sim_y v^x_m$, $I(Fy, u) = 0$. Therefore, this model and assignment is a counterexample to the argument.

\item
Let $M = \langle D, I \rangle$ be the model such that
\begin{center}
\begin{tblr}{
  width=0.5\textwidth,
  colspec={X[c]X[c]},
}
  $D = \{ m, n\}$
  &
  \begin{tblr}{c|cc}
    $I(R)$ & m & n\\
    \hline[solid]
    m & $0$ & $1$ \\
    n & $0$ & $1$
  \end{tblr}
\end{tblr}
\end{center}

This model provides a counterexample to the argument. The reasoning is represented in a diagram.
\begin{center}
\begin{tikzpicture}[
  >=stealth',
  node distance=1mm and 8mm,
  world/.style={
    ellipse,
    draw=black!66,
    fill=black!10,
    thick,
    minimum width=2.7cm,
    minimum height=1.2cm,
    font=\small
  },
  every label/.style={
    Maroon,
    font=\footnotesize
  },
  form/.style={font=\footnotesize}
]
\node [world] (v_1) [label=above:$v_1$] {x : m, y : n};
\node [world] (v_2) [label=above:$v_2$, right= 25mm of v_1] {x : m, y : m};
\node [world] (v_3) [label=above:$v_3$, below= 30mm of v_1] {x : n, y : n};
\node [world] (v_4) [label=above:$v_4$, right= 25mm of v_3] {x : n, y : m};

\node [form] (v_1_1) [below=of v_1] {$Rxy$};
\node [form] (v_2_1) [below=of v_2] {$\neg Rxy$};
\node [form] (v_3_1) [below=of v_3] {$Rxy$};
\node [form] (v_4_1) [below=of v_4] {$\neg Rxy$};

\node [form] (v_1_2) [below=of v_1_1] {$\exists y Rxy$, $\exists x Rxy$};
\node [form] (v_2_2) [below=of v_2_1] {$\exists y Rxy$, $\neg \exists x Rxy$};
\node [form] (v_3_2) [below=of v_3_1] {$\exists y Rxy$, $\exists x Rxy$};
\node [form] (v_4_2) [below=of v_4_1] {$\exists y Rxy$, $\neg \exists x Rxy$};

\node [form] (v_1_3) [below=of v_1_2] {$\forall x \exists y Rxy$, $\neg \forall y \exists x Rxy$};
\node [form] (v_2_3) [below=of v_2_2] {$\forall x \exists y Rxy$, $\neg \forall y \exists x Rxy$};
\node [form] (v_3_3) [below=of v_3_2] {$\forall x \exists y Rxy$, $\neg \forall y \exists x Rxy$};
\node [form] (v_4_3) [below=of v_4_2] {$\forall x \exists y Rxy$, $\neg \forall y \exists x Rxy$};
\end{tikzpicture}
\end{center}

\textcolor{red}{How to articulate this?}

\item

\end{enumerate}
\newpage