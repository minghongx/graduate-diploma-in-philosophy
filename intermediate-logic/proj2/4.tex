\section*{QUESTION 4}
The whole project seems to gradually shift the spotlight to the Barcan Formula (BF), implying that BF is the core of the controversy. Nonetheless, from readings one and two, it is understood that BF merely reflects one aspect of the controversy of Actualism. Another inference's validity, the Converse Barcan Formula (CBF), is more controversial than BF. This is because if CBF holds, it would deduce a logically impossible conclusion that there could be something that is distinct from itself. This suggests that the real focus might be on the logical challenges of Actualism. However, through reading three, it is found that the focus is not just on the debate, either metaphysical or logical, between Possibilism and Actualism, but more on Simple Quantified Modal Logic (SQML). The wording in Option 1 also hints that we are expected to defend a type of SQML, CDQS5, but to merely explain the metaphysical validity of BF.

In reading three, although authors claim that they are Possibilists, they take the position of Actualism in that paper. They propose a new semantic called contingently unconcrete, trying to repair the elegant SQML for Actualism, and hope to stop the continuous development and patching of ugly new logics in response to their logical challenges. This strange act of helping the opponent reveals that the authors' interests are not in which metaphysical stance but in SQML. The title of the article is not about defending Possibilism or Actualism, but SQML. That is to say, their real interest is in promoting SQML.

Possibilism is primarily concerned with SQML and does not care much about the validity of BF, and might even not be concerned about whether Possibilism is justified in metaphysics. For Possibilists, whether BF is valid or not is merely a choice. They can easily make BF valid by modifying semantics. For instance, Possibilists propose that the existence quantifier could be read as existence unloaded, or, as in reading three, consider some objects as contingently unconcrete. However, so far, the means by which Possibilism protects SQML inevitably need to appeal to possibilia, a concept opposed by Actualism. Possibilists rarely choose to invalidate BF, as it seems challenging to find an SQML that satisfies this condition, contradicting their pursuit of simple and elegant logic.

Since Actualism insists that the existence quantifier is existence loaded, they must reject BF. Rejecting BF forces them to seek a logic system beyond SQML. The logic proposed by Kripke revealed the feasibility of what they were pursuing. Kripke's Quantified Modal Logic, using varying domains and restricted quantifiers, solved three logical challenges faced by Actualism: BF, CBF, and NE, though it still use the concept of possibilia. Even with such a logical revelation, Actualism has not yet found a consistent logic system for their thesis.

Actualism's stance also seems incompatible with current developments in social metaphysics. Social metaphysics proposes social kinds, which exist within networks of social practices. As the network structure changes, existing social kinds may disappear, and nonexistent social kinds may emerge. For example, some social realities of the medieval period no longer exist today, and modern concepts of gender are being created. In this view, social kinds are a form of possibilia. Actualism's thesis would also deny social metaphysics, so not only in logic, but in metaphysics as well, Actualism faces severe challenges.

The successful interpretation of SQML by Possibilists, the various unsuccessful responses made by Actualism to logical challenges like BF, and the challenges faced by Actualism in contemporary metaphysics development all seem to hint at the triumph of SQML. However, this situation might only be temporary. We do not have sufficient arguments to prove that Actualism's claims have failed, nor enough evidence to justify the metaphysical correctness of Possibilism. As long as Actualism has not completely failed, they still hold the hope of finding a consistent logic for themselves. Logicians without a clear metaphysical stance, like the authors in reading three, can attempt to help Actualism find such logic, even though they appreciate the elegance of SQML more. Perhaps this is the motivation for setting the option 2 for this question.
