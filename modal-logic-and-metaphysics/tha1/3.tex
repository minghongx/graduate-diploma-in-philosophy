\section*{QUESTION 3}
\begin{enumerate}[label=\alph*)]
\item
The first degree is attaching ``$\bx$'' as a metalinguistic predicate to names of statements (a statement may have different formulation).
\[ \bx\ ``9 > 5'' \]
The second degree is \textit{de dicto} modality; The symbol ``$\bx$'' is attached to the statements themselves. This conflates metalanguage and object language.
\[ \bx\ 9 > 5 \]
The third degree is \textit{de re} modality. The symbol ``$\bx$'' is attached to a formula contains free occurrence of variable $x$:
\[ \bx\ x > 5 \]
This means that at the third degree, modalisers can appear within the scope of quantification. This is distinct from the second degree, where modalisers can only appear outside the scope of quantification, and thus can only modify closed formulae.

\item
Quine's argument against modalities as features of objects is tied to his rejection of essentialism — the idea that objects have some essential properties that define their identity across all possible worlds. For Quine, there are no necessary or essential properties of objects in this sense; rather, all properties are contingent upon the ways in which we choose to describe the world.

\item
\textit{de re} modality does not imply any essential properties. It just suggest that discussing in such terms is logical.

\end{enumerate}