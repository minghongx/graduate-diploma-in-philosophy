\begin{enumerate}[label=\alph*)]
\item \leavevmode \vspace{-1.75\baselineskip}

\begin{claim}
$\exists x Fx, \exists x Gx \not\vDash \exists x (Fx \land Gx)$
\end{claim}

\begin{proof}
Let $M = \langle D, I \rangle$ be the model such that $D = \{m, n\}$ and $I$ defined as
\begin{center}
\begin{tblr}{
  colspec={c|c|c},
}
& $I(F)$ & $I(G)$ \\
\hline[solid]
$m$ & $1$ & $0$ \\
$n$ & $0$ & $1$ \\
\end{tblr}
\end{center}
Let $v$ be an assignment whose two x-variant are $v[x : m]$ and $v[x : n]$. This provides a counterexample to $\exists x Fx, \exists x Gx \Yright \exists x (Fx \land Gx)$.

The assignment $v[x : m]$ yields
\begin{align*}
\begin{split}
I(Fx,v[x : m]) &= I(F)(I(x,v[x : m])) \\
&= I(F)(m) \\
&= 1
\end{split} \\
I(Gx,v[x : m]) &= \cdots = 0
\end{align*}

The assignment $v[x : n]$ yields
\begin{align*}
I(Fx,v[x : n]) &= \cdots = 0 \\
I(Gx,v[x : n]) &= \cdots = 1
\end{align*}

$I(\exists x Fx, v) = 1$, since $I(Fx,v[x : m]) = 1$. $I(\exists x Gx, v) = 1$, since $I(Gx,v[x : n]) = 1$. Thus, $I(A, v) = 1$ for each $A \in \{ \exists x Fx, \exists x Gx \}$, i.e., the premises are \textit{true}.

Since $I(Fx,v[x : m]) = 1$ and $I(Gx,v[x : m]) = 0$, $I(Fx \land Gx, v[x : m]) = 0$. Similarly, $I(Fx \land Gx, v[x : n]) = 0$. Thus, $I(\exists x (Fx \land Gx), v) = 0$, i.e., the conclusion is \textit{false}.
\end{proof}

\item[c)] This is a paradox. This is a logical truth

\end{enumerate}