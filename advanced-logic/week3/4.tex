Formula (b) entails formula (a) but not vice versa because (a) is a subformula of (b).

Formula (d) entails formula (b). Formula (d) reads as, there exists an object with properties F and G, and there exists an object with property F but without G. An object cannot simultaneously have and not have property G, so this means that the two objects with property F cannot be the same object.

However, formula (b) does not entail formula (d) because two objects that both have property F but are not identical can also both have property G.

Formula (c) means that if there exists an object with property F, then this object is unique. This does not specify whether there exists an object with property F, so it has no connection with formula (a). The uniqueness of an object with property F conflicts with formulas (b) and (d), as they imply the existence of at least two objects with property F.
